\documentclass[11pt,a4paper,oneside]{amsart}
\pdfoutput=1

%%%%%%%%%%%%%%%%%%%%%%%%% PACKAGES %%%%%%%%%%%%%%%%%%%%%%%%%%%%%%%

%\usepackage{refcheck}
\usepackage[shortlabels]{enumitem}
\usepackage[dvipsnames]{xcolor}
%\usepackage[alphabetic,initials]{amsrefs}
\usepackage{comment}
%\usepackage[pagebackref=true]{hyperref}
\usepackage[normalem]{ulem}
\usepackage{tcolorbox}
\usepackage{tikz-cd}
\usepackage{tcolorbox}
\usepackage{mathtools}
\usepackage{bm}
\usepackage{fancyhdr}
\usepackage{amsthm,amsmath,amscd,amssymb}
\pagestyle{fancy}
\usepackage[shortlabels]{enumitem}
\usepackage{breqn} 
\usepackage{amsfonts}
%\usetikzlibrary{calc}
%\usetikzlibrary{decorations.pathmorphing}
% A TikZ style for curved arrows of a fixed height, due to AndréC.
%\tikzset{curve/.style={settings={#1},to path={(\tikztostart)
 %   .. controls ($(\tikztostart)!\pv{pos}!(\tikztotarget)!\pv{height}!270:(\tikztotarget)$)
%    and ($(\tikztostart)!1-\pv{pos}!(\tikztotarget)!\pv{height}!270:(\tikztotarget)$)
    %.. (\tikztotarget)\tikztonodes}},
   % settings/.code={\tikzset{quiver/.cd,#1}
  %      \def\pv##1{\pgfkeysvalueof{/tikz/quiver/##1}}},
 %   quiver/.cd,pos/.initial=0.35,height/.initial=0}

% TikZ arrowhead/tail styles.
%\tikzset{tail reversed/.code={\pgfsetarrowsstart{tikzcd to}}}
%\tikzset{2tail/.code={\pgfsetarrowsstart{Implies[reversed]}}}
%\tikzset{2tail reversed/.code={\pgfsetarrowsstart{Implies}}}
% TikZ arrow styles.
%\tikzset{no body/.style={/tikz/dash pattern=on 0 off 1mm}}
%\usepackage[arrow,curve,matrix,tips,2cell]{xy}
\usepackage[hmargin=2cm,vmargin=2cm]{geometry}

\usepackage[backend=biber]{biblatex}
\addbibresource{ref.bib}


\usepackage{hyperref}
\usepackage{cleveref}


%%%%%%%%%%%%%%%%%%%%%%%%% FORMATTING %%%%%%%%%%%%%%%%%%%%%%%%%%%%%%%


\lhead{}
\chead{}
\rhead{}
\lfoot{}
\cfoot{\thepage}
\rfoot{}
\renewcommand{\headrulewidth}{0pt}
\renewcommand{\footrulewidth}{0pt}

\setlength{\headheight}{13pt}
\setlength{\footskip}{13pt}

%%%%%%%%%%%%%%%%%%%%%%%%% SPECIAL MACROS %%%%%%%%%%%%%%%%%%%%%%%%%%%%%%%


%To add label to inline equation; from https://tex.stackexchange.com/questions/271062/labeling-a-text-and-referencing-it-later
\makeatletter
\newcommand*{\inlineequation}[2][]{%
  \begingroup
    % Put \refstepcounter at the beginning, because
    % package `hyperref' sets the anchor here.
    \refstepcounter{equation}%
    \ifx\\#1\\%
    \else
      \label{#1}%
    \fi
    % prevent line breaks inside equation
    \relpenalty=10000 %
    \binoppenalty=10000 %
    \ensuremath{%
      % \displaystyle % larger fractions, ...
      #2%
    }%
    ~\@eqnnum
  \endgroup
}
\makeatother




%
%
\newtheorem{theorem}{Theorem}[section]
\newtheorem{exercise}{Exercise}[section]
\newtheorem{conjecture}{Conjecture}[section]
\newtheorem*{theorem*}{Theorem}
\newtheorem{lemma}[theorem]{Lemma}
\newtheorem*{lemma*}{Lemma}
\newtheorem{corollary}[theorem]{Corollary}
\newtheorem{proposition}[theorem]{Proposition}
\newtheorem{remark}[theorem]{Remark}
\newtheorem{definition}[theorem]{Definition}
\newtheorem*{definition*}{Definition}
\newtheorem*{definitions*}{Definitions}

\newtheorem{question}[theorem]{Question}
\newtheorem*{question*}{Question}
\newtheorem{questions}[theorem]{Questions}
\newtheorem*{questions*}{Questions}
\newtheorem{problem}[theorem]{Problem}

\newtheorem{example}[theorem]{Example}
\newtheorem{examples}[theorem]{Examples}

\newtheorem{thm}{Theorem}[subsection]
\renewcommand{\thethm}{\Alph{thm}}
\newtheorem{cor}[thm]{Corollary}
\renewcommand{\thecor}{\Alph{cor}}

%%%%%%%%%%%%%%%%%%%%%%%%% FONTS %%%%%%%%%%%%%%%%%%%%%%%%%%%%%%%


\newcommand{\mc}{\mathcal}
%
\newcommand{\cA}{\mathcal A}
\newcommand{\cB}{\mathcal B}
\newcommand{\cC}{\mathcal C}
\newcommand{\cD}{\mathcal D}
\newcommand{\cE}{\mathcal E}
\newcommand{\cF}{\mathcal F}
\newcommand{\cG}{\mathcal G}
\newcommand{\cH}{\mathcal H}
\newcommand{\cI}{\mathcal I}
\newcommand{\cJ}{\mathcal J}
\newcommand{\cK}{\mathcal K}
\newcommand{\cL}{\mathcal L}
\newcommand{\cM}{\mathcal M}
\newcommand{\cN}{\mathcal N}
\newcommand{\cO}{\mathcal O}
\newcommand{\cP}{\mathcal P}
\newcommand{\cQ}{\mathcal Q}
\newcommand{\cR}{\mathcal R}
\newcommand{\cS}{\mathcal S}
\newcommand{\cT}{\mathcal T}
\newcommand{\cU}{\mathcal U}
\newcommand{\cV}{\mathcal V}
\newcommand{\cW}{\mathcal W}
\newcommand{\cX}{\mathfrak{X}}
\newcommand{\cZ}{\mathcal Z}

\renewcommand{\H}{\textup{H}}

%
% doppelter Balken
%
\def\Az{\mathbb{A}}
\def\Cz{\mathbb{C}}
\def\Fz{\mathbb{F}}
\def\Iz{\mathbb{I}}
\def\Kz{\mathbb{K}}
\def\Lz{\mathbb{L}}
\def\Mz{\mathbb{M}}
\def\Nz{\mathbb{N}}
\def\Hz{\mathbb{H}}
\def\Pz{\mathbb{P}}
\def\Qz{\mathbb{Q}}
\def\Rz{\mathbb{R}}
\def\Tz{\mathbb{T}}
\def\Zz{\mathbb{Z}}
\def\Sz{\mathbb{S}}
\def\1z{\mathbb{1}}
%
% Fraktur
%
\newcommand{\fA}{\mathfrak A}
\newcommand{\fB}{\mathfrak B}
\newcommand{\fC}{\mathfrak C}
\newcommand{\fD}{\mathfrak D}
\newcommand{\fE}{\mathfrak E}
\newcommand{\fF}{\mathfrak F}
\newcommand{\fG}{\mathfrak G}
\newcommand{\fH}{\mathfrak H}
\newcommand{\fI}{\mathfrak I}
\newcommand{\fJ}{\mathfrak J}
\newcommand{\fK}{\mathfrak K}
\newcommand{\fL}{\mathfrak L}
\newcommand{\fM}{\mathfrak M}
\newcommand{\fN}{\mathfrak N}
\newcommand{\fP}{\mathfrak P}
\newcommand{\fT}{\mathfrak T}
\newcommand{\fW}{\mathfrak W}
\newcommand{\fX}{\mathfrak X}
\newcommand{\fY}{\mathfrak Y}


\newcommand{\mfa}{\mathfrak{a}}
\newcommand{\mfb}{\mathfrak{b}}
\newcommand{\mfc}{\mathfrak{c}}
\newcommand{\mff}{\mathfrak{f}}
\newcommand{\mfg}{\mathfrak{g}}
\newcommand{\mfk}{\mathfrak{k}}
\newcommand{\mfl}{\mathfrak{l}}
\newcommand{\mfm}{\mathfrak{m}}
\newcommand{\mfn}{\mathfrak{n}}
\newcommand{\p}{\mathfrak{p}}
\newcommand{\mfq}{\mathfrak{q}}
\newcommand{\mfs}{\mathfrak{s}}
\newcommand{\mft}{\mathfrak{t}}


\newcommand{\scA}{\mathscr{A}}
\newcommand{\scB}{\mathscr{B}}
\newcommand{\scS}{\mathscr{S}}
\newcommand{\scT}{\mathscr{T}}%


\newcommand{\eps}{\varepsilon}

%%%%%%%%%%%%%%%%%%%%%%%%% TEXTUP STUFF %%%%%%%%%%%%%%%%%%%%%%%%%%%%%%%


\newcommand{\fix}{\textup{fix}}

\newcommand{\res}{{\rm res}}
\newcommand{\Cl}{\textup{Cl}}
\newcommand{\Gal}{\textup{Gal}}
\newcommand{\Prim}{\textup{Prim}}
\newcommand{\Spl}{\textup{Spl}}
\newcommand{\sgn}{\textup{sign}}

\newcommand{\Int}{\int\limits}
\newcommand{\halb}{\tfrac{1}{2}}
\newcommand{\Hom}{{\rm Hom}\,}
\newcommand{\End}{{\rm End}\,}
\newcommand{\Aut}{{\rm Aut}}
\newcommand{\Tr}{{\rm Tr\,}}
\newcommand{\Sp}{{\rm Sp\,}}
\newcommand{\Spec}{{\rm Spec\,}} % Spektrum
\newcommand{\ind}{{\rm ind\,}}
\newcommand{\Coker}{{\rm Coker}\,}
\newcommand{\id}{{\rm id}}
\newcommand{\alg}{{\rm alg}}
\newcommand{\de}{{\rm deg\,}} % Grad
\newcommand{\tei}{\mid} % teilt
\newcommand{\ntei}{\nmid} % teilt nicht
\newcommand{\dis}{\displaystyle}
\newcommand{\Trace}{{\mathrm Trace}\,}
\newcommand{\Ind}{\mathrm{ Ind}\,}
\newcommand{\Sign}{\mathrm{ Sign}\,}
\newcommand{\Ad}{{\rm Ad\,}}
\newcommand{\lca}{{\rm LCA}}
\newcommand{\ab}{{\rm Ab}}
\renewcommand{\dim}{{\rm dim}}
\newcommand{\rk}{{\rm rk}}
\newcommand{\img}{{\rm im\,}}
\renewcommand{\ker}{{\rm ker}\,}
\newcommand{\coker}{{\rm coker}\,}
\newcommand{\lcm}{{\rm lcm}} % kleinstes gemeinsames Vielfaches
\newcommand{\reg}{^\times} % regulaer
\newcommand{\pos}{_{>0}} % positiv
\newcommand{\co}{^{\complement}} % Komplement
\newcommand{\tr}{\operatorname{tr}} % Spur
\newcommand{\ev}{\operatorname{ev}} % Auswertung
\newcommand{\lspan}{{\rm span}} % linearer Aufspann
\newcommand{\clspan}{\overline{\lspan}} % Abschluss des linearen Auspanns
\newcommand{\Iso}{{\rm Isom}} % Isometrien
\newcommand{\abs}[1]{\lvert#1\rvert} % Betrag
\newcommand{\norm}[1]{\left\|#1\right\|} % Norm

\newcommand{\supp}{{\rm supp}} % Support
\newcommand{\Min}{{\rm Min}} % Minimalpolynom


 \newcommand{\mat}[1]{\left( \begin{smallmatrix} #1 \end{smallmatrix}\right)} % small (inline) matrix
\newcommand{\acts}{\curvearrowright} 

\def\clmg{{\rm Cl}_\m^{\bar{\Gamma}}}
\def\clnl{{\rm Cl}_\n^{\bar{\Lambda}}}
\def\rmg{R_{\m,\Gamma}}
\def\pmg{P_{\m,\Gamma}}
\def\tor{{\rm tor}}
\def\Kmg{K(\m)^{\bar{\Gamma}}}
\def\Km{K(\m)}
\def\Knl{K(\n)^{\bar{\Lambda}}}
\def\Kn{K(\n)}

\newcommand{\ord}{\textup{ord}} 
\newcommand{\spn}{\textup{span}}
\renewcommand{\Im}{\textup{Im}}
\newcommand{\diag}{\textup{diag}}
\newcommand{\asc}{\textup{asc}}
\newcommand{\ann}{\textup{ann}}
\newcommand{\spec}{\textup{spec}}

% % % commands for full groups % % %
\newcommand{\fg}[1]{[\![#1]\!]}
\newcommand{\fu}{\mathfrak{u}}
\newcommand{\ag}[1]{\mathsf{A}(#1)}
\newcommand{\sg}[1]{\mathsf{S}(#1)}
\newcommand{\Sym}{\textrm{Sym}}
\newcommand{\Homeo}{\textrm{Homeo}}
\newcommand{\Isom}{\textup{Isom}}
\newcommand{\Sub}{\textup{Sub}}
\newcommand{\ol}[1]{\overline{#1}}
\newcommand{\GL}{\textup{GL}}
\newcommand{\SL}{\textup{SL}}
\newcommand{\M}{\textup{M}}
\newcommand{\Mult}{\textup{Mult}}
\newcommand{\kdim}{\textup{kdim}}
%\newcommand{\alg}{\textup{alg}}
\newcommand{\gp}[1]{\langle #1\rangle}
\newcommand{\mon}[1]{\langle #1\rangle^+}

\newcommand{\Stab}{\textup{Stab}}
\newcommand{\dom}{\textup{dom}}
\newcommand{\ran}{\textup{ran}}
%\newcommand{\tor}{\textup{tor}}
%
%
\newcommand{\im}{\textup{im}}
\newcommand{\msc}{\mathscr}
\newcommand{\bd}{\partial\widehat{\cE}}
\newcommand{\E}{\widehat{\cE}}
\renewcommand{\sp}{\textup{sp}}

\newcommand{\bbt}{\mathbbm{t}}
\newcommand{\F}{\mathbf{F}}


\newcommand{\blue}{\textcolor{blue}}
\newcommand{\red}{\textcolor{red}}
\newcommand{\green}{\textcolor{green}}

\newcommand{\affT}[1][A] {\mathrm{Aff \thinspace T(#1)}}
\newcommand{\algk}[1][A]{\overline K^{alg}_1(\mathrm{#1})}

\newcommand{\rrr}[1][3]{\mathbb{R}^#1}

\begin{document}

\title{The Evans-Kishimoto intertwining}

\thispagestyle{fancy}



\author{Carlos Campoy}



\date{\today}


\maketitle


\parskip=0em

\setlength{\parindent}{0cm} \setlength{\parskip}{0.5cm}


\maketitle

\section{Introduction}

This is a short note explaining how to apply the Evans-Kishimoto intertwining of two $*$-automorphisms $\alpha$ and $\beta$ with the \textit{Rokhlin property}. This notes are just a reformulation of Gábor Szabó's notes given as supplementary material for a 3-hour lectures series presented at the 16th Spring Institute for Non-commutative Geometry and Operator Algebras (NCGOA) from the 14th to the 19th of May 2018. Some parts we will skip and refer the reader to these notes which can be found in  \href{https://gaborszabo.nfshost.com/publications.html}{https://gaborszabo.nfshost.com/publications.html}. Other sections will be explained more thoroughly. I hope will help those -like me- who could not attend the lecture series. For actions of \textit{locally compact groups} on $C^*$-algebras there is a philosophy underlying many results which can be summarised in the following recipe.



\begin{tcolorbox}[width=17cm, colframe=black, colback=Cyan! 20, halign=left]
\underline{\textit{Recipe:}} Suppose we have two group actions $\alpha, \beta: G\acts A$ which we want to show they are \textit{cocycle conjugate}. We wish to \begin{enumerate}
\item Show that $\alpha$ and $\beta$ satisfy some sort of \textit{Rokhlin property}. 
\item Using the first step one wants something like $\Ad(w_g)\circ \alpha_g \approx \beta_g$ holds approximately in point-norm over a large finite set in $A$ and uniformly over a compact set in $G$. Do the same in the reverse direction. 
\item Show that both $\alpha$ and $\beta$ has the \textit{approximately central cohomology vanishing property} i..e, for any $\alpha$-cocycle (and $\beta$-cocycle) $w$ with $\norm {[a,w_g]}\approx 0$ over some finite set find a unitary $v$ with $\norm{[v,a]}\approx 0$ and $w_g\approx v\alpha_g(v^*)$. 
\item Apply the \textit{Evans-Kishimoto intertwining} technique. 
\end{enumerate}
\end{tcolorbox}

In these notes we will explain the ingredients of the recipe in the case of single automorphisms (equivalently $\mathbb Z$-actions) with the $\textit{Rokhlin property}$  for unital $C^*$-algebras\footnote{This is just for simplicity.} with a few more (simplifying) assumptions we will make. Let us start from the beginning. 

\section{Rokhlin automorphisms}

\begin{tcolorbox}[colback= LimeGreen! 50]
\begin{definition}
Let $A$ and $B$ be $C^*$-algebras equipped with automorphisms $\alpha\in \mathrm{Aut(A)}$, and $\beta\in \mathrm{Aut(B)}$. we say $\alpha$ and $\beta$ are cocycle conjugate if there exists $w\in \mathcal U(\mathcal M(A))$ and an isomorphism $\varphi: A\rightarrow B$ such that $$\Ad w\circ \alpha= \varphi^{-1}\circ \beta \circ \varphi.$$ 
\end{definition}
\end{tcolorbox}

\begin{tcolorbox}[colback= LimeGreen! 50]
\begin{definition}
Let $A$ be unital and $\alpha: A\rightarrow A$ be a $*$-automorphism. A unitary $u\in \mathcal U(A)$ is a co-boundary if there exists $v\in \mathcal U(A)$ with $u=v\alpha(v^*)$. 
\end{definition} 
\end{tcolorbox}
The next is our most important definition. 

\begin{tcolorbox}[colback= LimeGreen! 50]
\begin{definition}
\label{Rokklin prop}
Let $A$ be unital and separable and $\alpha \in \mathrm{Aut(A)}$. We say $\alpha$ has the Rokhlin property if for all $n\in \mathbb{N}$ there exists approximately central sequences of projections $(e_k)$ and $(f_k)$ in $A$ such that $$1=\lim_{k} \sum_{j=0}^{n-1}\alpha^j(e_k)+\sum_{l=0}^{n}\alpha^l(f_k).$$
\end{definition}
\end{tcolorbox}

\begin{remark}
Thinking in terms of the central sequence algebra $A_\infty\cap A'$ this means for every $n\in \mathbb{N}$  we can find projections $e,f\in A_\infty\cap A'$ such that $$1=\sum_{j=0}^{n-1}\alpha^j(e)+\sum_{l=0}^{n}\alpha^l(f).$$ Thus we have $2n+1$ pairwise orthogonal projections and moreover $\alpha^n(e)=e$ and $\alpha^{n+1}(f)=f$. So that $\alpha$ cycles around these orthogonal projections. 
\end{remark}

\begin{tcolorbox}[colback= LimeGreen! 50]
\begin{definition}
We say $A$ has the \underline{strict} Rokhlin property if for every $n\in \mathbb{N}$ we can find a projection $e\in A_\infty\cap A'$ with $$1=\sum_{j=0}^{n-1}\alpha^{j}(e).$$
\end{definition}
\end{tcolorbox}
\begin{remark}
It is highly non-trivial to see when $C^*$-algebras can have an automorphism with the Rokhlin property specially since it requires the existence of non-trivial projections. For example $C_r^*(\mathbb{F}_n)$ does not have any projections, or $C(X)$ where $X$ is compact and connected. 
\end{remark}

To guarantee \textit{approximately central cohomology vanishing} we will impose a very strong condition on our $C^*$-algebras. 

\begin{tcolorbox}[colback= LimeGreen! 50]
\begin{definition}
\label{prop (*)}
Let $A$ be a unital $C^*$-algebra. We say that $A$ has property $(*)$ if there exists $L>0$ such that for all $\varepsilon>0$ and $\mathcal F\subset\subset A$ there exists $\delta>0$ and $\mathcal G \subset \subset A$ such that if $u\in U(A)$ and we have $$\mathrm{max}_{a\in \mathcal G}\norm{[a,u]}\leq \delta$$ there exists an $L$-Lipschitz path $u_t$, with $u_0=1$ and $u_1=u$, such that $$\mathrm{max}_{a\in \mathcal F}\mathrm{max}_{t\in I}\norm{[a,u_t]}<\varepsilon.$$
\end{definition}
\end{tcolorbox}

\begin{remark}
In terms of sequence algebras, this means $A_\infty\cap A'$ is connected (through $L$-Lipschitz paths (?)).
\end{remark}

\begin{theorem}
\label{UHF have (*)}
All UHF-algebras satisfy property $(*)$. 
\end{theorem}
\begin{proof}
Let $\mathbb U$ be a UHF-algebra and let $L=\pi$. Fix $\varepsilon>0$ and some finite set $\mathcal F\subset \subset A$. Because of the structure of UHF algebras we have $\mathbb U\cong M_p\otimes \mathbb U_1$ such that $p\in \mathbb N$ can be chosen to approximate $\mathcal F$ as well as we want. Thus we may assume $\mathcal F\subset M_p\otimes 1$ w.l.o.g. Now, choose $\mathcal G=\{e_{ij}\otimes 1\}$ i..e, the units in $M_p$ diagonally included. Notice that we have a conditional expectation $$E: \mathbb U\rightarrow 1_p\otimes \mathbb U_1: a\mapsto \sum_{i\leq p}(e_{ii}\otimes 1) a (e_{ii}\otimes 1)$$ such that if $ \mathrm{max}_{b\in \mathcal G}[a, b]\leq \delta$, we have $\norm{E(a)-a}\leq p\delta$. Moreover, we can thus fix $\delta>0$ such that that if a unitary $u$ $\delta$-commutes with $\mathcal G$ then there exists a $v\in 1\otimes U_1$, which can be chosen with finite spectrum as close as we want from $u$. Moreover choosing an appropriate branch of the logarithm we can connect $u$ to $v$ through $\gamma_t=e^{log(uv^*)t}v$ which can be chosen to be $\pi$-Lipschitz and of length $\frac{\varepsilon}{2 \mathrm{max_{a\in \mathcal F}\{\norm a\}}}$ or $\varepsilon/2$ if $F=\{0\}$. Since $v$ has finite spectrum it is connected to $1$ through via a $\pi$-Lipschitz path. Hence the concatenation of these two paths gives our result as the second commutes with $\mathcal F$ and $$[\gamma_t, a]\leq \norm{\gamma_t a- va + va - a\gamma_t+av -av}\leq \epsilon$$ as desired.
\end{proof}

\begin{lemma}
\label{approx}
Let $A$ be unital, separable and with property $(*)$. Then, we have that if $\alpha$ is a Rokhlin automorphism on $A$, for all $u\in U(A_\infty\cap A')$ there exists $v\in U(A_\infty\cap A')$ such that $u=v\alpha(v^*)$ i..e, an (approximately central) co-boundary. \footnote{This is sometimes refer to as the \textit{one-cocycle property}.}\footnote{We will use the finite $(\varepsilon-\delta)$-version of this result.}
\end{lemma}
\begin{proof}
First notice that given a sequence of approximately central sequences $(v_k^{(n)})_k$, the diagonal sequence i.e., $(v^{(k)}_k)_k$ is also approximately central. As a result, it is enough to proof lemma \ref{approx} up to $\varepsilon$ i..e, we show that for all $\epsilon>0$ and for all $u\in U(A_\infty\cap A')$ there exists $v\in U(A_\infty\cap A)$ with $u\approx_\varepsilon v\alpha(v^*)$. First, let $\varepsilon>0$ and pick $n\in \mathbb{N}$ such that $L/n<\varepsilon$, where $L>0$ is the constant from property $(*)$ (definition \ref{prop (*)}). Then we apply the Rokhlin property (definition \ref{Rokklin prop}) to that same $n\in \mathbb{N}$ so that there exists $e,f\in A_\infty\cap A'$ such that the projections $\alpha^j(e)$, $\alpha^l(f)$ partition the space. Moreover, choose a $L$-Lipschitz paths $z_t$, $\gamma_t$ with $z_0=1$ and $z_1=\alpha^n(u_n^*)$, and $\gamma_0$, $\gamma_1=\alpha^{n+1}(u_{n+1}^*$ where $u_k=u\cdots \alpha^{k-1}(u)$\footnote{This choice is made because $u_j\alpha(u_{j-1})=u$ and $\alpha^n(z^*_{\frac{n-1}{n}})\approx u_n$}. Notice we can change the projections $e$ and $f$ by taking sub-sequences and diagonals so that they commute with elements of the form $\alpha^k(z_t)$ and $\alpha^k(u)$ for all $k\in \mathbb{Z}$ and rational $t$ since this set of elements is countable, which will imply the projections commute with $\alpha^k(z_t)$ for all $k\in \mathbb{Z}$ and $t\in [0,1]$. We repeat the same process with $\gamma_t$. Consequently $$v=\sum_{j=0}^{n-1}\alpha^j(e)u_j\alpha^j(z_{j/n})+\sum_{l=0}^{n}\alpha^l(f)u_l\alpha^l(\gamma_{\frac{l}{n+1}})$$ is a unitary such that since $alpha$ cycles the projections, and all projections are orthogonal \begin{align*}v\alpha(v^*)= e \alpha^n(z^*_{\frac{n-1}{n}})\alpha(u^*_{n-1})& +\sum_{j=1}^{n-1}\alpha^j(e)u_j\alpha^j(z_{j/n}z^*_{\frac{j-1}{n}})\alpha(u^*_{j-1})\\& + f\alpha^{n+1}(\gamma^*_{\frac{n}{n+1}})\alpha(u^*_{n+1})+\sum_{l=1}^{n}\alpha^j(f)u_l\alpha^l(\gamma_{\frac{l}{n+1}}\gamma^*_{\frac{l-1}{n+1}})\alpha(u^*_{l-1})\\& \approx_\varepsilon \sum_{j=1}^{n}\alpha^j(e)u_j\alpha(u_{j-1}^*)+\sum_{l=1}^{n+1}\alpha^l(f)u_l\alpha(u_{l-1}^*)\\&= u\end{align*} since $\alpha^n(z^*_{\frac{n-1}{n}})\approx_\epsilon u_n$, $\alpha^{n+1}(\gamma^*_{\frac{n}{n+1}})\approx_\varepsilon u_{n+1}$, $\alpha^l(\gamma_{\frac{l}{l+1}}\gamma^*_{\frac{l-1}{n+1}})\approx_{\varepsilon} 1$, and $\alpha^j(z_{j/n}z^*_{\frac{j}-1{n}})\approx_\varepsilon 1$ and assuming without loss of generality that $A_\infty\cap A'\subset \mathcal B(\mathcal H)$ we see that $$\mathcal H= \bigoplus_{j=1}^{n}\alpha^j(e)(\mathcal H)\oplus\bigoplus_{l=1}^{n+1}\alpha^l(f)(\mathcal H)$$ and each summand above maps a direct summand to itself. Then we use that an operator of the form $a=\oplus_{i}u_i$ has norm given by $\sup_i\norm{u_i}$.
\end{proof}


\begin{remark}
\label{remark of approx vanishing}
In terms of finite sets this means that for all $\varepsilon>0$ and $\mathcal F\subset \subset A$ there exists $\delta>0$ and $\mathcal G\subset\subset A$ such that given a unitary $u\in \mathcal U(A)$ if $$\mathrm{max}_{a\in \mathcal G}\norm{[u,a]}\leq \delta$$ then there exists $v\in \mathcal U(A)$ such that for all $a\in \mathcal F$ $$\mathrm{max}\{\norm{u-v\alpha(v^*)}, [a,v]\}\leq \varepsilon.$$   


\end{remark}
\begin{definition}
If a unital and separable $C^*$-algebra $A$ equipped with an (not necessarily Rokhlin) automorphism $\alpha\in \mathrm{Aut(A)}$ satisfies the conclusions of lemma \ref{approx} we say the pair $(A,\alpha)$ has the one-cocycle property or simply $\alpha$ has the one-cocycle property when the $C^*$-algebra in question is fixed. Moreover, we say $(A,\alpha)$ has the inner  one-cocycle propery if $(A, \Ad w\circ \alpha)$ has the one co-cycle property for all $u\in \mathcal U(A)$ 
\end{definition}



\section{The Evans-Kishimoto interwining}





In this section we discuss the final step of our recipe. Notice for now we have restricted our attention to single automorphisms which are equivalent to $\mathbb{Z}$-actions. We will soon be looking at $\mathbb{R}$-actions, where we will be able to adapt some of the discussed methods. In such instance the endgame will again be an Evans-Kishimoto type argument. 


we will now state the following theorem which will be the focus of the next section. The proof of the theorem will be the content of the \textit{Evans-Kishimoto intertwining}. 

\begin{theorem}
Let $A$ be separable and unital, and assume $\alpha$ and $\beta$ are automorphisms with the inner one-cocycle property. Then, we have that $\alpha\approx_{au}\beta$ if and only if $\alpha$ and $\beta$ are cocycle conjugate via an approximately inner automorphism. \label{Evans-Kishimoto} 
\end{theorem}

\begin{question}
Is the one-cocycle property closed under conjugation? If not, can we find a big class of $\mathbb{Z}-C^*$-algebras with this property much bigger than those with property $(*)$?
\end{question}



\begin{tcolorbox}[colback= Cyan! 20] 
\underline{The ideas behind the proof}. We will combine $\alpha \approx_{au}\beta$ with the one-cocycle property for these maps and any conjugation by unitaries $\Ad w\circ \alpha$ and $\Ad u\circ \beta$.  We take an incresing sequence of finite sets $\mathcal F_n$ whose union is dense and we will use the unitaries  we get from the equivalence to create new automorphisms $\alpha_{2k}$, $\beta_{2k+1}$ which agree more and more as $k$ goes to infinity. This will give as sequence of automorphisms with \begin{align}Ad u_{2(k-1)}\cdots \Ad u_0\circ \alpha \approx \Ad u_{2k-1}\cdots \Ad u_1\circ \beta\end{align} and by a clever zig-zag argument we will produce with lemma \ref{approx} an approximately central sequence of unitaries $(v_n)$ such that \begin{align}
u_{2k} \approx v_{2k}\alpha(v_{2k}^*)
\end{align} and \begin{align}
u_{2k+1}\approx v_{2k+1}\alpha(v_{2k+1}).
\end{align} The $v_n$s can be chosen in such a way that the approximate centrality implies that the sequence of automorphisms $\Ad v_{2k}\cdots v_0$ and $\Ad v_{2k+1}\cdots v_1$ converge, to (approximately inner) automorphisms $\varphi_0$ and $\varphi_1$ respectively.  Moreover, we will ensure 
\begin{align}
\varphi_0\circ\alpha\circ\varphi_0^{-1} \end{align} is a correction by unitaries away from \begin{align}
\varphi_1\circ\beta\circ\varphi_1^{-1}.
\end{align} 

More concretely, we use $(1)$ and notice that the sequence $$X_n=\begin{cases} v_{2k}\cdots v_0 (u_{2k}\cdots u_0) \alpha(v_{2k}\cdots v_0)& \text{ if } n=2k \\v_{2k+1}\cdots v_1 (u_{2k+1}\cdots u_1) \alpha(v_{2k+1}\cdots v_1)& \text{ if } n=2k+1\end{cases}$$ gives the correcting unitaries, which we will need to make sure it converges (at least the subsequences of odd and even terms). So that $$\varphi_0\circ \Ad  \lim_k X_{2k} \circ \alpha \circ \varphi_0^{-1}= \varphi_1\circ \Ad \lim_k X_{2k+1} \circ \beta \circ \varphi_1^{-1}$$ which would give the desired result.
\end{tcolorbox}

\begin{proof}[Proof of theorem \ref{Evans-Kishimoto}] The if statement is clear so we focus on the only if direction. So we assume $\alpha \approx_{au} \beta$ and choose a sequence of finite sets $\mathcal F_n$ in the unit ball with dense union. We now start an induction process. \begin{enumerate}
\item[(Step 1)] By lemma \ref{approx} for $\beta$ given the pair $(1/2, \mathcal F_1)$ there exists a pair $(\delta_1, \mathcal G_1)$ and now define $$\mathcal G^{'}_1= \beta^{-1}(\mathcal G_1)\cup \mathcal F_1^{'}$$ and choose $\mathcal F_1^{'}:= \mathcal F_1$, and pick $\delta_1\leq 1/2$. Now choose $u_0\in \mathcal U (A)$ such that  $$\mathrm{max}_ {a\in \mathcal G_1^{'}}\norm {\beta(a)-u_0\alpha(a)u_0^*}\leq \delta_1/2.$$ Now define $\alpha_2:= \Ad u_0 \circ \alpha$, $v_0:= u_0$,  $\mathcal F_2^{'}= \mathcal F_2\cup \Ad v_0(\mathcal F_2)$ and for $alpha_2$ take $(2^{-2}, \mathcal F_2^{'})$ take $(\delta_2, \mathcal G_2)$ with $\delta_2\leq \mathrm{min}\{2^{-2}, \delta_1\}$. Thus, taking $\mathcal G_2^{'}=\alpha_2^{-1}(\mathcal G_2)\cup \mathcal F_2^{' }$ we can find $u_1\in \mathcal U (A)$ such that $$\mathrm{max}_{a\in \mathcal G_2^{'}}\norm {\alpha_2(a)-u_1\beta_1(a)u_1^*}\leq \delta_1$$

and notice this implies $$\mathrm{max}_{a\in \mathcal G_1}\norm{[u_1, a]}$$ so that by lemma \ref{approx} there exists $v_1\in \mathcal U(A)$ such that for all $a\in \mathcal F_1^{'}$ $$\norm{u_1-v\beta_1(v_1)}\leq 2^{-1}$$ and $$\norm{[a,v]}\leq 2^{-1}$$

\item[(Step 2)] Set $\beta_3= \Ad u_1\circ \beta_1$ and $\mathcal F_3^{'}=\mathcal F_3\cup \Ad v_1(\mathcal F_3)$ and choose for the pair $(2^{-3}, \mathcal F_3^{'})$ a pair $(\delta_3, \mathcal G_3)$ with $\delta_3\leq \mathrm{min}\{2^{-3}, \delta_2\}$ using lemma \ref{approx}. Define $\mathcal G_3 ^{'}= \beta_3^{-1}(\mathcal G_3 ^{'})\cup \mathcal F_3 ^{'}$ and pick $u_2$ with $$\norm{\beta_3(a)- u_2\alpha_2(a)u_2^*}\leq \delta_3/2$$. As before, this implies $$\mathrm{max}_{a\in \mathcal G_2}\norm{[a,u_2]}\leq \delta_2$$ and by lemma \ref{approx} we can find $v_2\in \mathcal U(A)$ such that for all $a\in \mathcal F_2^{'}$ we have $$\norm{u_2-v_2\alpha(v_2^*)}\leq 2^{-2}$$ and $$\norm{[a, v_2]}\leq 2^{-2}.$$

Inductively we can now find unitaries $u_k, v_k \in \mathcal U(A)$ such that 

\begin{align}
\alpha_{2(k+1)}=\Ad u_ {2k}\circ \alpha_{2k},
\end{align}  
\begin{align}
\beta_{2k+3}=\Ad (u_{2k+1})\circ \beta_{2k+1},
\end{align}
\begin{align}
\mathrm{max}_{a\in \mathcal F_{2k}}\norm{\beta_{2k+1}(a)-\alpha_{2k}(a)}\leq 2^{-(2k+1)},
\end{align}
\begin{align}
\norm{u_{2k}-v_{2k}\alpha_{2k}(v_{2k}^*)}\leq 2^{-2k}, \text{ and } \norm{u_{2k+1}- v_{2k+1}\beta_{2k+1}(v_{2k+1}^*)}\leq 2 ^{-(2k+1)},
\end{align}
\begin{align}
\mathrm{max}_{a\in \mathcal F_n^{'}}\norm{[v_n,a]}\leq 2^{-n},
\end{align}
\begin{align}
\mathcal F_{2k}^{'}=\mathcal F_{2k}\cup \Ad(v_{2(k-1)\cdots v_0})(\mathcal F_{2k}), \text{ and } \mathcal F_{2k+1}^{'}= \mathcal F_{2k+1}\cup \Ad(v_{2k-1}\cdots v_1)(\mathcal F_{2k+1}). 
\end{align}

Combining this equations we can actually execute the idea of the proof word by word. Combining $(10)$ and $(11)$ we see the sequences of $*$-automorphisms $\Ad (v_{2k}\cdots v_0)$ and $\Ad (v_{2k+1}\cdots v_1)$ are point-wise Cauchy for all $a\in \mathcal F_n$, and $n\in \mathbb{N}$. Thus, they must me Cauchy for all $a\in A$. Consequently the sequences converge to automorphisms $\varphi_0$ and $\varphi_1$. We finish with a short computation showing that $X_{2k}$ and $X_{2k+1}$ as defined before are both Cauchy since then we can copy-paste the final steps of the idea of proof.      

Notice, $$\norm{u_{2k}-v_{2k}\alpha_{2k}(v_{2k}^*)}\leq 2^{2k}$$and this implies$$\norm{v^*_{2k}u_{2k}\alpha_{2k}(v_{2k})-1}\leq 2^{-2k}.$$ Consequently, letting $V_{2k}=v_{2k}\cdots v_0$ and $U_{2k}= u_{2k}\cdots u_0$ for simplicity, we see


\begin{align*}
X_{2k}& =V^*_{2k}U_{2k}\alpha(V_2k)\\ &= V^*_{2(k-1)}v^*_{2k}u_{2k}\alpha_{2k}(v_{2k}V_{2(k-1)})U_{2k-1}\\& \approx_{2^{-2k}} V^*_{2(k-1)}\alpha_{2k}(V_{2(k-1)})U_{2(k-1)}=X_{2(k-1)}.
\end{align*}
The same trick works for odd terms now using we have$$\norm{u_{2k+1}-v_{2k+1}\beta_{2k+1}(v_{2k+1}^*)}\leq 2^{-(2k+1)}$$and thus$$\norm{v_{2k+1}^*u_{2k+1}\beta_{2k+1}(v_{2k+1})-1}\leq 2^{-(2k+1)}$$ so that $$X_{2k+1}\approx_{2^{-(2k+1)}} X_{2k-1}$$ and the result follows. 
\end{enumerate}

\end{proof}

\section{The categorical approach: A possible alternative to Evans-Kishimoto}


\section{The Classification of Rokhlin flows}


This section is a synthesis of \cite{szabo2021classification} which we will later make use of quite heavily.

\begin{definition}
Let $\alpha: \mathbb{R}\acts A$ be a group action. We define the map $$\alpha_{co}^T: A\rightarrow C([0,T], A): a\mapsto (\alpha_{co}^T: t\mapsto \alpha_t(a))$$
for all $T>0$. In particular, we let $\alpha_{co}:=\alpha_{co}^1$. 
\end{definition}


\begin{lemma}
Let $\alpha,\beta: \mathbb{R}\acts A$ be two group actions. Then, the following are equivalent.

\begin{enumerate}
\item $\alpha_{co}\approx_u \beta_{co}$, 
\item $\alpha^T_{co}\approx_u \beta_{co}^T$, 
\item $\alpha^T_{co}\approx_u \beta_{co}^T$ implemented by unitaries in $1+ \mathcal C_0((0,T], A)$. 
\end{enumerate}
\end{lemma}

\begin{proof}
Notice that $(3)\implies (2)\implies (1)$ so it is enough to show $(1)\implies (3)$. Moreover notice that if for some $T>0$ we see $\alpha_{co}^T\approx_u \beta_{co}^{T}$; then,  $\alpha_{co}^{T'}\approx_u \beta_{co}^{T'}$ for all $T'\leq T$. Consequently, it is enough to show $(1)\implies (3)$ for all $T\in \mathbb{N}-\{0\}$. So, fix $T\geq 1$, let $\mathcal F\subset \subset A$, and pick $\varepsilon>0$. Since both actions are continuous, we can define the compact sets \begin{align}
\mathcal F'=\bigcup_{0\leq s \leq T-1}\beta_s(\mathcal F)
\end{align} 
\begin{align}
\label{compactset_1}
\mathcal F''=\bigcup_{0\leq s\leq 1}\beta_s(\mathcal F')
\end{align}

and notice $\mathcal F'\subseteq \mathcal F''$. We use this compact sets to define a unitary $w\in 1+\mathcal C_0((0,T], A)$ with the desired properties. First, take $w'\in \mathcal U(1+C(I, A))$ such that for all $a\in \mathcal F''$ 

\begin{align}
\label{compactset_2}
\norm{\Ad (w')\circ \alpha_{co}(a)-\beta_{co}(a)}=\max_{s\in I}\norm{\Ad (w_s')\circ \alpha_s(a)-\beta_s(a)}\leq \varepsilon/{2T} 
\end{align}  

Choosing $s=0$ and by (\ref{compactset_2}), we have that (abusing notation) for all $s\in \mathcal F'$
\begin{align}
\label{compactset_3}
\max_{a\in \mathcal F'}\max_{s\in I}\norm{w'_0\alpha_s(a){w_0'}^{*}-\alpha_s(a)}\leq \epsilon/{2T}. 
\end{align} 

Consequently, we can put these results together to get a unitary $w=w'{w_0'}^*$ with $w_0=1$ so that 
\begin{align}
\label{nicerintertwining}
\max_{a\in \mathcal F'}\max_{s\in I}\norm{\Ad w_s\circ \alpha_s(a)-\beta_s(a)}\leq \epsilon/L, \text{ and } w\in 1+\mathcal C_0((0,1], A).
\end{align}

We can extend this unitary to $1+ \mathcal C_0((0,T], A)$ inductively, and we will show it witnesses the approximate unitary equivalence between $\alpha_{co}^T$ and $\beta_{co}^T$ for $\mathcal F \subset \subset A$. Thus, for $j\in \{0,..., T-1\}$ and $s\in I$ define \[w^{(T)}_{j+s}=w^{(T)}_j\alpha_j(w_s).\] Now, we set a decreasing sequence from $\mathcal F'$ to $\mathcal F$ given by \[\mathcal F'_k=\bigcup_{0\leq s\leq T-1-k}\beta_s(\mathcal F).\] Besides, it is worth noting $\beta_s(\mathcal F'_{k+1})\subseteq \mathcal F'_{k}$ for all $s\in I$. Now, we will show inductively that for $k\leq T-1$ we have \begin{align}
\max_{a\in \mathcal F'_k} \max_{s\in I}\norm {\Ad w_{k+s}^{(T)}\circ \alpha_{k+s}(a)-\beta_{k+s}(a)}\leq {(k+1)}\varepsilon/T
\end{align}
which clearly implies $\Ad w^{(T)}\circ \alpha_{co}(a)\approx_\varepsilon \beta_{co}(a)$ for all $a\in \mathcal F$ since $ \mathcal F\subseteq \mathcal F'_k$ and any $t\in [0,T]$ can be written as $k+s$ for some $k\leq T-1$ and $s\in I$. Notice that for $k=0$ we get (\ref{compactset_3}). Now assume true for $k<T-1$ and notice for $s=1$ the induction hypothesis gives 

\begin{align}
\label{IH_1}
\max_{a\in \mathcal F'_k} \norm {\Ad w_{k+1}^{(T)}\circ \alpha_{k+1}(a)-\beta_{k+1}(a)}\leq {(k+1)}\varepsilon/T.
\end{align}

For $a\in \mathcal F'_{k+1}$, $s\in I$ we see 

\begin{align}
 \norm {\Ad w_{k+1+s}^{(T)}\circ \alpha_{k+1+s}(a)-\beta_{k+1+s}(a)}&= \norm{\Ad (w^{(T)}_{k+1}\alpha_{k+1}(w_s))\circ \alpha_{k+1}(\alpha_s(a)-\beta_{k+1}(\beta_s(a))}\\& = \norm{\Ad w_{k+1}^{(T)}(\alpha_{k+1}(\Ad w_s(\alpha_s(a))))- \beta_{k+1}(\beta_s(a))}\\& \leq \varepsilon/T + \norm{\Ad w^{(T)}_{k+1}(\alpha_{k+1}(\beta_s(a)))-\beta_{k+1}(\beta_s(a))}\\ & \leq (k+2)\varepsilon/T.
\end{align}

To go from the second to the third line we use (\ref{nicerintertwining}) and from the third to the forth we use (\ref{IH_1}). This completes the proof. 
\end{proof}

\begin{definition}
Let $\alpha,\beta:\Rz A$ be two flows. We say $\alpha$ and $\beta$ are cocycle conjugate if there exists an $\alpha$-cocyle $w: \Rz\rightarrow \mathcal M(\mathcal U (A))$ and $\varphi\in \Aut(A)$ such that for all $t\in \Rz$ \[\Ad w_t\circ \alpha=\varphi^{-1}\circ \beta_t\circ \varphi\] and we  write $\alpha \sim_\varphi \beta$. 
\end{definition}

\begin{lemma}
Let $\alpha,\beta:\mathbb{R}\acts A$ where $A$ is separable. If $\alpha\sim_{\varphi} \beta$ and $\varphi \in \overline {\mathrm{Inn}}(A)$, then $\alpha_{co}\approx_u \beta_{co}$. 
\end{lemma}

\begin{proof}
Notice it is enough to show the proposition for exterior equivalence and conjugacy (with inner automorphism) separately. \begin{itemize}
\item If $\alpha_t=\varphi^{-1}\circ\beta_t\circ \varphi$. Fix $(\mathcal F, \varepsilon)$ then we can choose (by compactness and using that $A$ is separable) a unitary $u\in \mathcal U(1+A)$ such that for all $a\in \mathcal F$  \[\max\{\norm{\varphi(a)-\Ad(u)(a)}, \max_{s\in I}\norm{\varphi^{-1}(\alpha_t(\varphi(a)))-\Ad u^*(\alpha_t(\varphi(a)))}\}\leq \varepsilon/2.\] The co-boundary $w: t\mapsto w_t=u^*\alpha_t(u)$ now does the job since \[\norm{\Ad w\circ \alpha_{co}(a)-\beta_{co}(a)}=\max_{s\in I}\norm{ u^*\alpha_s(u)\alpha_s(a)\alpha_s(u^*)u-\beta_s(a)}= \norm{\Ad u^*\circ \alpha_s\circ \Ad u(a)-\beta_s(a)}\]
a quick computation gives \[\norm{\Ad w\circ \alpha_{co}(a)-\beta_{co}(a)}\approx_{\varepsilon/2}  \norm{\Ad u^*\circ \alpha_s\circ \varphi(a)-\beta_s(a)} \approx_{\varepsilon/2}0\]
for all $a\in \mathcal F$. Since $\varepsilon>0$ was arbitrary the result follows. 

\item If $\alpha$ and $\beta$ are exterior equivalent there exists a $\alpha$-cocycle $w: \mathbb{R}\rightarrow (\mathcal U(\mathcal M(A)))$ such that \[\Ad w_t\circ \alpha_t=\beta_t.\] If $A$ is unital we are done by choosing the cocycle. Otherwise, we use a result due to Kashimoto \cite[Theorem 1.1]{kishimoto2006multiplier} which states that given a $p\in A$, $\alpha$-cocycle $u_t$ we can find an $\alpha$-cocycle $v_t$ with values in $\mathcal U(1+A)$ such that $\norm{(u_t-v_t)}\leq \varepsilon$ for $\varepsilon>0$ and $t\in [-1,1]$ which (find how) can be extended to finite sets $\mathcal F\subset \subset A$. Consequently, for a pair $(\mathcal F, \varepsilon)$ we can find an $\alpha$-cocycle $v_t$ with values in $\mathcal U(1+A)$ such that

$$\norm{\Ad v_t\circ \alpha_t- \Ad u_t\circ \alpha_t}\leq \varepsilon$$ and viewing the cocycle as a unitary  $v\in 1+C([0,1], A)$ we have \[\max_{a\in \mathcal F}\norm{\Ad v\circ \alpha_{co}(a)-\beta_{co}(a)}\leq \varepsilon\] as desired.


\end{itemize}
\end{proof}
\begin{lemma}
Kishimoto's \cite[Theorem 1.1]{kishimoto2006multiplier} implies that for all finite $\mathcal F\subset \subset A$ given a cocycle $u:\mathbb{R}\rightarrow \mathcal M(\mathcal U(A))$, and for all $varepsilon>0$ we can find a cocycle $v:\mathbb{R}\rightarrow \mathcal U(1+A)$ such that \[\max_{a\in \mathcal F}\norm{\Ad v_t \alpha_t(a)- \Ad u_t\alpha_t}\leq \varepsilon \text{ for all } t\in I.\]

\end{lemma}


\begin{proof}

Given $\varepsilon>0$ apply \cite[Theorem 1.1]{kishimoto2006multiplier} to $A^n$ given $\mathcal F=\{a_1,...,a_n\}$ and $\underline{a}=\{a_1,...,a_n\}\in A^n$ so that given a cocycle $u$ with values in $\mathcal M(\mathcal U(A))$, we can consider $\underline u:=\oplus_{i\leq n} u$ and we can find and $\oplus_{i\leq n}\alpha$- we abuse notation and still call it $\alpha$- cocycle $v$ as above such that \[\norm{u_t(\alpha_t(\underline a)-v_t \alpha_t(\underline a)}\leq \varepsilon/4\] since $a\mapsto a^*$ is continuous we can choose $v$ such that \[\norm{\alpha_t(\underline a) \underline u_t^*-\alpha_t(\underline a) v^*_t}\leq \varepsilon/4.\] Consequently, we see that \[\norm{\Ad v_t \alpha_t(\underline a)- \Ad u_t \alpha_t(\underline a)}=\norm{\Ad v_t \alpha_t(\underline a)- v_t\alpha_t(\underline a)u_t^*- \Ad u_t \alpha_t(\underline a)+v_t\alpha_t(\underline a)u^*_t}\leq \varepsilon/2 \]

by the triangle inequality. Now, since $\underline u$ is diagonal (idea does not work) 
\end{proof}



\section{The rotation map}




\section{Rokhlin Flows on non-commutative tori}

In this section we want to explore how Rokhlin flows can be constructed on non-commutative tori as first discussed by Kishimoto \cite{kishimoto1996rokhlin}. We first try to understand how to make it work for irrational rotation algebras $A_{\theta}$ 
we will often make use the following identification 


 \[A_\theta\cong \mathcal C (\mathbb T)\rtimes_{\alpha^\theta} \mathbb Z\]

where the action $\alpha^{\theta}: \mathbb Z\acts \mathbb T$ is given by $\alpha^\theta_1(z)=e^{-2\pi i \theta}z$ for a fixed $\theta\in \mathbb I$. More, generally $A_\theta$ can be understood as the universal $C^*$-algebra generated by two unitaries $u_1, u_2$ such that $$u_1u_2=u_2u_1e^{2\pi i \theta}$$. it is sometimes refer to as the non-commutative tori, and one can analogously define  non-commutative $n$-tori as the universal $C^*$-algebra generated by unitaries $u_1,..., u_n$ with commutation relations \[u_iu_j=u_ju_ie^{2\pi i \theta_{ij}}\] where $ \vartheta=(\theta_{ij})_{ij}$ is an anti-symmetric matrix i.e., $\vartheta^T=-\vartheta$. To understand the examples in this section we will need the following results about closed subgroups of $\mathbb{R}^n$. The proofs of these results are not hard but they are quite ingenious. They are all taken from \cite[Chapter VII]{bourbaki1966elements}.

\begin{lemma}
Let $G$ be a closed subgroup of $\mathbb{R}^n$ which is not discrete. Then it contains a line i.e., there exists $a\in G-\{0\}$ with $ta\in G$ for all $t\in \mathbb R$.
\end{lemma}

\begin{corollary}
The vector space $V$ generated by the union of all lines contained in $G$ is a non-empty subgroup of $G$.  
\end{corollary}

\begin{corollary}
$G$ is the direct sum of $V$ and $V^\perp\cap G$, where $V^\perp\cap G$ does not contain any line passing through the origin so it is discrete.  
\end{corollary}

\begin{corollary}
Since any discrete group $G$ of $\mathbb R^n$ is finitely generated, we have that any closed subgroup is of the form $\mathbb{R}^a\times \mathbb{Z}^b\times \{0\}^{n-(a+b)}$ up to a transformation by an invertible matrix i.e., \[G\cong A(\mathbb{R}^a\times \mathbb{Z}^b\times \{0\}^{n-(a+b)})\] for some $A\in GL_n(\mathbb{R})$. 
\end{corollary}

\begin{corollary}
Given a closed subgroup $G\subseteq \mathbb{R}^n$ we can define the group \[G^*:=\{y\in \mathbb{R}^n: x\cdot y=0 \text{ for all } x\in G\}\] by the previous results $$G\mapsto G^*$$ defines an inverse Galois correspondence for the lattice of subgroups of $\mathbb R^n$ to itself. In particular, $G^*=0$ if and only if $G=\mathbb{R}^n$.
\end{corollary}

\begin{proposition}\cite[Proposition 2.5]{kishimoto1996rokhlin} 
\label{Roklin property tori} Let $A(\vartheta)$ be a non-commutative n-torus which $\vartheta \mathbb{Z}^n\cap \mathbb{Z}^n=\{0\}$ such $p=(p_1,..., p_n)^T\in \mathbb R^n$ with $\mathbb{Z}^n+\vartheta \mathbb{Z}^n \cap \mathbb{R}p=\{0\}$. Then the flow given by $$\alpha_t(u_j)=e^{2\pi i p_it}u_j$$ has the Rokhlin property. \end{proposition}

To prove proposition \ref{Roklin property tori} we use the following 

\begin{lemma}
The set \[\mathcal S=\{(\vartheta m+k), p\cdot m\};\] where $\vartheta$ and $p\neq 0$, are as above, and $m,k\in \mathbb{Z}^n$, is dense. 

\end{lemma}

\begin{proof}
Note that $\overline G$ is a closed subgroup of $\mathbb R^n\times \mathbb R$. Assuming $\overline{G}\neq \mathbb{R}^n\times \mathbb{R}$ we see $G^*$ contains a non-zero element $(\zeta_0, \zeta_1)\in \mathbb{R}^n\times \mathbb{R}$. Consequently, we see that for all $k,m\in \mathbb{Z}^n$ \[\langle \zeta_0,k \rangle-\langle \vartheta\zeta_0,  m\rangle+ \langle \zeta_1p, m\rangle \in \mathbb{Z} \text{ for all } m,k\in \mathbb{Z}^n\]

thus, we see that $\zeta_0\in \mathbb{Z}^n$ choosing $m=0$ and $k=e_j$. Now, choosing $k=0$ and $m=e_j$ we can conclude that \[\zeta_1p-\vartheta\zeta_0=w\in \mathbb{Z}^n;\] from here it follows that $w+\vartheta \zeta_0\in \mathbb{R}p\cap (\mathbb{Z}^n+ \vartheta \mathbb{Z}^n)=0$ i.e., $\vartheta \zeta_0=w$ which implies by our assumption that $\zeta_0=0$. Thus, $$\zeta_1 p\in \mathbb{R}p\cap (\mathbb{Z}^n+ \vartheta \mathbb{Z}^n)=0$$ from which is clear that $\zeta_1=0$. This yields a contradiction as $(\zeta_0, \zeta_1)\neq 0$.
\end{proof}

\section{Questions}
\begin{question}
Why do we use $\mathcal U(1+A)$ in the notes?
\end{question}

\begin{question}{How much weaker is the inner one-cocycle property than the combination of $(*)$ and the Rokhlin property?}
\end{question}

\begin{question}
Differences between the generality of Evans-Kishimoto type arguments versus the two sided intertwining in \cite{szabo2021categorical}
\end{question}


\printbibliography

\end{document}
